%%
%% This is file `sample-sigconf.tex',
%% generated with the docstrip utility.
%%
%% The original source files were:
%%
%% samples.dtx  (with options: `sigconf')
%% 
%% IMPORTANT NOTICE:
%% 
%% For the copyright see the source file.
%% 
%% Any modified versions of this file must be renamed
%% with new filenames distinct from sample-sigconf.tex.
%% 
%% For distribution of the original source see the terms
%% for copying and modification in the file samples.dtx.
%% 
%% This generated file may be distributed as long as the
%% original source files, as listed above, are part of the
%% same distribution. (The sources need not necessarily be
%% in the same archive or directory.)
%%
%% The first command in your LaTeX source must be the \documentclass command.
\documentclass[sigconf]{acmart}
\settopmatter{printacmref=false}
\renewcommand\footnotetextcopyrightpermission[1]{}
%%
%% \BibTeX command to typeset BibTeX logo in the docs
\AtBeginDocument{%
  \providecommand\BibTeX{{%
    \normalfont B\kern-0.5em{\scshape i\kern-0.25em b}\kern-0.8em\TeX}}}



%%
%% Submission ID.
%% Use this when submitting an article to a sponsored event. You'll
%% receive a unique submission ID from the organizers
%% of the event, and this ID should be used as the parameter to this command.
%%\acmSubmissionID{123-A56-BU3}

%%
%% The majority of ACM publications use numbered citations and
%% references.  The command \citestyle{authoryear} switches to the
%% "author year" style.
%%
%% If you are preparing content for an event
%% sponsored by ACM SIGGRAPH, you must use the "author year" style of
%% citations and references.
%% Uncommenting
%% the next command will enable that style.
%%\citestyle{acmauthoryear}

%%
%% end of the preamble, start of the body of the document source.
\begin{document}

%%
%% The "title" command has an optional parameter,
%% allowing the author to define a "short title" to be used in page headers.
\title{Parallelization on Minimal Dominating Set with Game Theory}

%%
%% The "author" command and its associated commands are used to define
%% the authors and their affiliations.
%% Of note is the shared affiliation of the first two authors, and the
%% "authornote" and "authornotemark" commands
%% used to denote shared contribution to the research.
\author{Hsin-En, Su 0616215}
\author{Po-Yu, Liu 0616223}
\author{Yu-Wei, Shih 0616213}



%%
%% By default, the full list of authors will be used in the page
%% headers. Often, this list is too long, and will overlap
%% other information printed in the page headers. This command allows
%% the author to define a more concise list
%% of authors' names for this purpose.
\renewcommand{\shortauthors}{Trovato and Tobin, et al.}



%%
%% This command processes the author and affiliation and title
%% information and builds the first part of the formatted document.
\maketitle

\section{Introduction}
Minimal dominating set(MDS) is a NP-complete problem, however there are several way to approximate the solution. A game-theoratic  approach has been proposed by Li-Hsing Yen [1] which allows the problem be solved in a distributed scenario. Also, they mentioned a synchronous daemon method but not implemented due to its complexity. Which inspired us to try and design a proper parallelized version of the algorithm.

\section{Statement of the Problem}
One special characteristic of the game theory is that it can have more than one solution known as Nash Equilibrium. The solution is depended on the sequence of the actions that players take in the game system. This resembles the characteristic that threads are also not determinisitc, but depends on the scheduling of the operating system. So we are interesting in benefits we might gain by mimicking players in the game system with threads.


\section{Proposed Approaches}
As mentioned above, we want to mimick the players in the game system with threads. We will create a thread for every node in the graph. To avoid simultaneos action between neighboring nodes, we give a backoff probability to each player to reduce the conflict rate. Note that having a conflict will not make the result wrong but only to slightly increase the computation time. Nodes that did not backoff will check the status of their neighbors and compute its utility, then make a decision to either set itself on or off. This process is repeated until the graph find a Nash Equilibirum, and no one continues to change their decision.

\begin{figure}[h]
	\includegraphics[scale=0.3]{image1.png}
\end{figure}
\section{Related Work}
Li-Hsin Yen's work[1] is highly related to our project, however our focus is mostly different. Their work is focusing on solving the problem with game theory, while our project is to solve the algorithm in parallel to boost the computation time on the NP-complete problem.

\section{Language Selection}
In this project, we choose OpenMP and Pthread for multi-threading methods and implement the project in C++. OpenMP is a convenient option in C++ 11 or later versions. However, we might require some of the flexibility of pthread due to the idea to mimick the players with thread, so we might use both of them to implement our algorithm. 



\section{Expected Result}
We expect that the computation time will be relatively short due to players can now simultaneously make decisions.


\section{Timetable}
\begin{tabular}[t]{lll}
	\hline
	Work & Deadline & Remarks\\
	\hline
	Reading paper & 11/11 & algorithm and paper \\
	Finish sequential code & 11/25 & maybe in C++ \\
	Finish parallel code & 12/9 & need to discuss before \\
	Optimize parallel & 12/30 & * \\
	Finish fianl report & 1/6 & need work division \\
	Finish research & 1/6 & yeah! \\
	\hline
\end{tabular}

\section{References}
[1] Li-Hsing Yen Member, IEEE and Zong-Long Chen. "Game-Theoretic Approach to Self-Stabilizing Distributed Formation of Minimal Multi-Dominating Sets"

[2] H. Kakugawa ; T. Masuzawa. "A self-stabilizing minimal dominating set algorithm with safe convergence"

[3] E. W. Dijkstra, “Self-stabilizing systems in spite of distributed
control"

[4] N. Guellati and H. Kheddouci, “A survey on self-stabilizing
algorithms for independence, domination, coloring, and matching
in graphs,"

[5] Z. Xu, S. T. Hedetniemi, W. Goddard, and P. K. Srimani, "A
synchronous self-stabilizing minimal domination protocol in an
arbitrary network graph,"

[6] N. Megiddo, "Applying Parallel Computation Algorithms in the Design of Serial Algorithms", 22nd Annual Symposium on Foundations of Computer Science, pp. 399-408, 1981-October.





\end{document}
\endinput
%%
%% End of file `sample-sigconf.tex'.
